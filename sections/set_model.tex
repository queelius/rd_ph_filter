\documentclass[ ../main.tex]{subfiles}
\providecommand{\mainx}{..}
\begin{document}
\section{Approximate and oblivious positive sets}
A set is given by the following definition.
\begin{definition}
A set is an unordered collection of distinct elements from a universe of elements.
\end{definition}
A countable set is a \emph{finite set} or a \emph{countably infinite set}. A \emph{finite set} has a finite number of elements. For example,
\[
    \St = \{ 1, 3, 5 \}
\]
is a finite set with three elements. A \emph{countably infinite set} can be put in one-to-one correspondence with the set of natural numbers. The cardinality of a set $\St$ is a measure of the number of elements in the set, denoted by
\begin{equation}
    \card{\mathbb{\St}}\,.
\end{equation}
The cardinality of a \emph{finite set} is a non-negative integer and counts the number of elements in the set, e.g.,
\[
    \card{\left\{ 1, 3, 5\right\}} = 3\,.
\]

The abstract data type of the immutable \emph{positive} approximate set\cite{aset} is given by the following definition.
\begin{definition}
\label{def:approx_set}
The abstract data type of the \emph{positive approximate set} over a countably infinite universe $\mathbb{U}$ has the following operations defined:
\begin{align}
    &\Contains \colon \Powerset(\mathbb{U}) \times \mathbb{U} \mapsto \{ \True, \False \}\,,\\
    &\FalsePositiveRate \colon \Powerset(\mathbb{U}) \mapsto (0,1)\,.
\end{align}
Let an element that is selected uniformly at random from the universe $\mathbb{U}$ be denoted by $\rv{X}$. A set $\Sp$ is a \emph{positive approximate set} of a set $\St$ with a false positive rate $\varepsilon$ if the following conditions hold:
\begin{enumerate}[(i)]
    \item $\St$ is a subset of $\Sp$. This condition guarantees that no \emph{false negatives} may occur.
    \item If $\rv{X}$ is \emph{not} a member of $\St$, it is a member of $\Sp$ with a probability $\varepsilon$,
    \begin{equation}
        \Pr\left[\Contains\!\left(\Sp,\rv{X}\right) \given \neg \Contains\!\left(\St,\rv{X}\right)\right] = \varepsilon\,.
    \end{equation}
    \item The false positive rate function returns $\varepsilon$,
    \begin{equation}
         \FalsePositiveRate(\Sp) = \varepsilon\,.
    \end{equation}
\end{enumerate}
\end{definition}

The optimal space complexity of \emph{countably infinite} positive approximate sets is given by the following postulate.
\begin{postulate}
The \emph{optimal} space complexity of a data structure implementing the \emph{positive approximate set} over \emph{countably infinite} universes is independent of the type of elements and depends only the false positive rate $\varepsilon$ as given by
\begin{equation}
    -\log_2 \varepsilon \; \si{bits \per element}\,.
\end{equation}
\end{postulate}

The immutable abstract data type of the \emph{oblivious set}\cite{obset} is given by the following definition.
\begin{definition}
\label{def:obliviousset}
Assuming that the only information about a set of interested $\St$ is given by another set $\Osp$, $\Osp$ is an \emph{oblivious positive set} of $\St$ with a false positive rate $\varepsilon$ if the following conditions hold:
\begin{enumerate}[(i)]
    \item $\Osp$ is a positive approximate set of $\St$ with a false positive rate $\varepsilon$.
    \item The false positives are uniformly distributed over $\mathbb{U} \setminus \St$.
    \item There is no efficient way to enumerate the elements in $\Osp$.\footnote{That is, the true positives and false positives.}
    \item Any estimator of the cardinality of $\St$ may only be able determine an approximate upper and lower bound, where the uncertainty may be traded for space-efficiency.
\end{enumerate}
\end{definition}

The absolute space efficiency of a data structure implementing an oblivious positive set consisting of $m$ true positives with a false positive rate $\varepsilon$ and an entropy $\beta$ is given by
\begin{equation}
    \AE(\varepsilon, m, \beta) = \expectation\!\left[\frac{-(m + \rv{X}) \log_2 \varepsilon}{\BL\!\left(\MakeObliviousSet(m + \rv{X},\varepsilon)\right)}\right]\,,
\end{equation}
where
\begin{equation}
    \rv{X} \sim \dudist(0,2^\beta - 1)\,.
\end{equation}

The relative efficiency of the \emph{optimal} positive oblivious set with entropy $\beta$ to the \emph{optimal} positive approximate set ($\beta = 0$) has an expectation given by
\begin{equation}
    \RE(\,\cdot\,, m, \beta) = 2^{-\beta} \sum_{k=0}^{2^\beta-1} \left(1 + \frac{k}{m}\right)^{-1}\,.
\end{equation}
For a fixed $\beta$, as $m \to \infty$ the relative efficiency goes to $1$.


\end{document}