\documentclass[ ../main.tex]{subfiles}
\providecommand{\mainx}{..}
\begin{document}
\section{Applications}
The \emph{approximate map} trades \emph{accuracy} for \emph{space efficiency}. Therefore, the stored representation of the keys is lossy which makes recovering a key from its representation impractical. Thus, the \emph{approximate map} facilitates the implementation of space and time efficient non-iterable approximate containers.

\subsection{Approximate multi-set}
A multi-set (bag) is given by the following definition.
\begin{definition}
A multi-set is an unordered collection of elements from a universe where each element may occur multiple times. More formally, a multi-set $\Set{S}$ is a group of elements with the following properties:
\begin{enumerate}
    \item All elements belong to a universe $\Set{U}$.
    \item For each $x \in \Set{U}$, $x$ is either a member of $\Set{S}$ or it is not. If $x$ is a member, it may occur multiple times, denoted the multiplicity of $x$.
    \item The elements in $\Set{S}$ are not ordered.
\end{enumerate}
\end{definition}
We are interested in the countable multi-sets. A countable multi-set is a \emph{finite multi-set} or a \emph{countably infinite multi-set}. A \emph{finite multi-set} has a finite number of elements. For example,
\[
    \{ 1, 3, 1, 1 \}
\]
is a finite set with two distinct elements, where element $1$ has a multiplicity of $3$ and element $3$ has a multiplicity of $1$. A \emph{countably infinite multi-set} can be put in one-to-one correspondence with the set of natural numbers.

The cardinality of a multi-set $\Set{A}$ is a measure of the number of elements in the multi-set, denoted by
\begin{equation}
    \Card{\Set{A}}\,.
\end{equation}
The cardinality of a \emph{finite multi-set} is a non-negative integer and counts the number of elements in the set, e.g.,
\[
    \Card{\left\{ 1, 3, 1, 1\right\}} = 4\,.
\]

The \gls{gls-pmf} may implement an approximate multi-set where the multiplicy of $x$ is given by
\begin{equation}
    \Mp[x]\,.
\end{equation}
The space complexity of a multi-set implemented using the \gls{gls-pmf} is given approximately by
\begin{equation}
    m \log_2 \!\left( \frac{n}{\fprate} \right) + 1.44 m\,.
\end{equation}
where $m$ is number of unique elements in the multi-set, $n$ is the largest multiplicity of any of the elements in the multi-set, and $\fprate$ is the false positive rate.

\subsection{Mutable approximate sets}
\label{sec:mutable}
In the approximate hash map, deleting an $x \in \PASet{S} \setminus \Set{S}$ will cause a false negative on an element $y \in \Set{S}$, where $\ph(y) = \ph(x)$.

After $k-1$ insertions have succeeded, inserting an $x \in \overline{\PASet{S}}$ succeeds with probability $1-r_k$ where $r_k$ is the load factor after $k$ successful insertions and is given by
\begin{equation}
    r_k = r_{k-1}\left(1 + \frac{1}{m_{k-1}}\right)\,,
\end{equation}
where $m_k$ is the cardinality of $\Set{S}$ after $k$ successful insertions with a base case given by $r_0$.

The false positive rate after $k$ successful insertions is given by
\begin{equation}
    \fprate = r_k 2^{-M}\,.
\end{equation}

Attempting to insert an $x \in \overline{\PASet{S}}$\footnote{For any $x \in \PASet{S}$, it is already a member.} into $\Set{S}$ may fail due to colliding with an existing member.
We could remember this outcome and say that the given $x$ is not a member of $\Set{S}$.
However, we take the approach of quantifying the outcome of an insertion probabilistically.

\begin{definition}
The discrete random variable $\RV{K} \in \{0, 1, \ldots\}$ denotes the number of successful insertions.
\end{definition}

Given a $\pmapf$ The probability mass function that $\RV{K} = k$ insertions succeed given $n$ insertion attempts and $t$ deletions is given by the recurrence relation
\begin{equation}
    \PDF{k \Given m_0, n, t}[\RV{K}] = \PDF{k-1 \Given n-1}[\RV{K}]\left(1 - r_{k-1}\right) + \PDF{k \Given n-1}[\RV{K}] r_k\,,
\end{equation}


The probability mass function that $\RV{K} = k$ insertions succeed given $n$ insertion attempts is given by the recurrence relation
\begin{equation}
    \PDF{k \Given n}[\RV{K}] = \PDF{k-1 \Given n-1}[\RV{K}]\left(1 - r_{k-1}\right) + \PDF{k \Given n-1}[\RV{K}] r_k\,,
\end{equation}
with a base case given by
\begin{equation}
    \PDF{k \Given 0}[\RV{K}] = \SetIndicator{k=0}\,.
\end{equation}

Since we know the distribution of successful inserts after $n$ insertions, we may compute the false negative rate that results from this.

A false negative occurs for an element $x$ whenever we insert $x$ and the insertion fails. If we try to insert $n$ elements not in $\PASet{S}$ (note that successfully inserting an element $x$ increases the false positive rate 

See \Cref{sec:guarded} to see how to make deletions and insertions deterministically succeed over a \emph{guarded} set.



%\input{sections/apps_postings_list}
\end{document}
