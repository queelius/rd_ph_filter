\documentclass[ ../main.tex]{subfiles}
\providecommand{\mainx}{..}
\begin{document}
\section{Introduction}
The Perfect Hash Filter is a data structure that is suitable to implement two types of sets, \emph{positive approximate sets}\cite{aset} and \emph{oblivious sets}\cite{obset}.
Informally, positive approximate sets generate false positives at specifiable rate and oblivious sets are a type of approximate set with additional \emph{confidentiality} guarantees.

In \cref{sec:set_model}, we precisely define concept of a set.
Then, we define the approximate and oblivious positive sets.
In \cref{sec:phf}, we derive the Perfect Hash Filter, a data structure that implements the approximate positive set and prove various properties, such as its absolute space efficiency.
Following from that, in \cref{dummyref} we show how to tweak the implementation of the Perfect Hash Filter\footnote{For instance, by varying the load factor.} to implement the \emph{oblivious set}. 
\end{document}